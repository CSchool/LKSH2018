В фирме, выпускающей компьютерные комплектующие, все изделия получают последовательные номера от 1 до $N$. Каждое изделие после его изготовления поступает в отдел контроля качества, где оно проверяется, и либо уходит в продажу, либо заносится в список бракованных изделий и списывается. К сожалению, список бракованных изделий иногда оказывается чересчур длинным. Тогда для его сокращения подряд идущие числа заменяются интервалом: через тире указываются номера первого и последнего изделия интервала. Например, вместо \t{1,3,4,5,6,7,8,10,12,16,17,20,21,22,23,24} записывается \t{1,3-8,10,12,16-17,20-24}.

Напишите программу, которая по полному списку номеров бракованных изделий, выдаст этот список в сокращенном виде. 
