\begin{problem}{Список}{стандартный ввод}{стандартный вывод}{1 секунда}{256 мегабайт}

В фирме, выпускающей компьютерные комплектующие, все изделия получают последовательные номера от 1 до $N$. Каждое изделие после его изготовления поступает в отдел контроля качества, где оно проверяется, и либо уходит в продажу, либо заносится в список бракованных изделий и списывается. К сожалению, список бракованных изделий иногда оказывается чересчур длинным. Тогда для его сокращения подряд идущие числа заменяются интервалом: через тире указываются номера первого и последнего изделия интервала. Например, вместо \t{1,3,4,5,6,7,8,10,12,16,17,20,21,22,23,24} записывается \t{1,3-8,10,12,16-17,20-24}.

Напишите программу, которая по полному списку номеров бракованных изделий, выдаст этот список в сокращенном виде. 


\InputFile
Вводится сначала число $N$ --- общее количество изделий. Затем число $M$ --- количество изделий, оказавшихся бракованными. Далее вводятся в возрастающем порядке номера бракованных изделий. $1 \le M \le N \le 10^5$


\OutputFile
Выведите в одной строке список номеров бракованных изделий в сокращенном виде. Интервалы должны разделяться запятой. В строке не должно быть пробелов.

\Examples

\begin{example}
\exmp{10 5
1 3 5 7 9
}{1,3,5,7,9
}%
\exmp{40 16
1 3 4 5 6 7 8 10 12 16 17 20 21 22 23 24
}{1,3-8,10,12,16-17,20-24
}%
\exmp{11 11
1 2 3 4 5 6 7 8 9 10 11
}{1-11
}%
\exmp{10000 1
5
}{5
}%
\end{example}

\end{problem}

