Большой любитель математики Вова решил повесить у себя в комнате таблицу умножения. После некоторых раздумий он обнаружил, что обычная таблица умножения 10 на 10 уже не популярна в наши дни. Он решил повесить у себя в комнате таблицу n на m. Представив себе эту таблицу, Вова задался вопросом - сколько раз в ней встречается каждая из цифр от 0 до 9?


И прежде чем нарисовать эту таблицу Вова попросил вас написать программу, которая даст ответ на его вопрос. 
Как известно, в таблице умножения на пересечении строки $i$ и столбца $j$ записано число $i \cdot j$. 