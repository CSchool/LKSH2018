\begin{problem}{Кузнечик}{стандартный ввод}{стандартный вывод}{1 секунда}{256 мегабайт}

Кузнечик Игорь любит прыгать по камням. Сегодня, 6 августа 2128 года, ровно 73 года и 4 месяца спустя Момента Апокалипсиса, он проснулся в отличном настроении и решил преодолеть $n$ камней, стоящих на пути от его домика до складов с радиоактивным плутонием. 

Кузнечик Игорь точно знает, что он может прыгать только на $k$ камней, то есть с $i$-го камня, он может перепрыгнуть на $i + 1, i + 2, \ldots, i + k$ камни. Он начинает свой путь с камня с номером $1$ и хочет добраться до камня с номером $n$.

Кузнечика Игоря интересует, сколько есть различных способов добраться до его пункта назначения. Ну а вы, конечно же, должны ему помочь, исключительно из доброты душевной, а не за какую-либо шоколадку.

\InputFile
На входе подаётся два целых числа: $n$ и $k$ ($1 \le k \le n \le 50$).

\OutputFile
Выведите одно целое число~---~ответ на задачу.

\Example

\begin{example}
\exmp{5 2
}{5
}%
\end{example}

\end{problem}

