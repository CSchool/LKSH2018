\begin{problem}{Цифры}{стандартный ввод}{стандартный вывод}{1 секунда}{256 мегабайт}

Чтобы привлечь самых маленьких детей в летнюю компьютерную школу <<КЭШ>> руководством было приобретено очень много красивых пластмассовых цифр. Однажды мимо коробки, где они хранились, проходил ди-джей Евгений. Заметив среди цифр нули он решил, что это сушки, и все их съел. Теперь детям на уроках математики приходится решать задачи без использования нулей. Например, иногда их просят составить любое число с суммой цифр равной N. Мы не спрашиваем, было ли у Евгения расстройство желудка. То, что нас интересует, это сколько разных чисел с суммой разрядов N можно составить из цифр от 1 до 9.

\InputFile
Число N от 1 до 29.

\OutputFile
Единственное число ---- количество чисел, которые можно составить из цифр, сумма которых равна N.

\Example

\begin{example}
\exmp{4
}{8
}%
\end{example}

\end{problem}

