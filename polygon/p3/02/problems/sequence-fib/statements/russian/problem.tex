\begin{problem}{Последовательность Фибоначчи}{стандартный ввод}{стандартный вывод}{1 секунда}{256 мегабайт}

$F_k$ --- бесконечная последовательность целых чисел, которая
удовлетворяет условию Фибоначчи $F_k = F_{k-1}+F_{k-2}$
(для любого целого $k$). Даны $i$, $F_i$, $j$,
$F_j$, $n$ ($i \ne j$). Найти
$F_n$. Пример части последовательности:
$F_{-2}=-5, F_{-1}=4, F_{0}=-1, F_{1}=3, F_{2}=2, F_{3}=5, F_{4}=7, F_{5}=12, F_{6}=19$

Ограничения: $-1000 < i, j, n < 1000$, 
$-2000000000 < F_k < 2000000000$
($k = min(i, j, n)\ldots max(i,j,n)$)

\InputFile
В первой строке находятся числа $i$, $F_i$, $j$, $F_j$, $n$.

\OutputFile
Вывести одно число $F_n$.

\Example

\begin{example}
\exmp{3 5 -1 4 5
}{12
}%
\end{example}

\end{problem}

