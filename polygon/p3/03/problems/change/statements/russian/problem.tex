\begin{problem}{Сдача}{стандартный ввод}{стандартный вывод}{1 секунда}{256 мегабайт}

Когда Миша и Маша покупали подарок, возникла интересная ситуация. У них была в
распоряжении только одна большая купюра, а у продавца~---~некоторое количество мелочи. Дело
происходило утром, поэтому продавцу нужно было экономить мелочь, и он хотел отдать сдачу
минимальным количеством монет. Подумав некоторое время, они точно определили, с каким
количеством монет продавцу придется расстаться. А вы сможете решить такую задачу?


\InputFile
В первой строке записано число $n$ ($1 \le n \le 10$)~---~количество различных
номиналов монет, содержащихся в кассе. Можно считать, что количество монет каждого номинала
достаточно.

На следующей строке содержится n целых чисел $a_i$ ($0 \le a_i \le 2000$)~---~номиналы монет.

В третьей строке записано одно число $k$ ($1 \le k \le 10^6$)~---~сумма, которую нужно набрать.


\OutputFile
Выведите минимальное количество монет, которое придется отдать продавцу,
или \t{-1}, если продавец вообще не сможет дать им сдачу.

\Examples

\begin{example}
\exmp{3
1 3 5
13
}{3
}%
\exmp{4
5 6 7 8
9
}{-1
}%
\end{example}

\end{problem}

