\begin{problem}{Ход конём}{стандартный ввод}{стандартный вывод}{1 секунда}{256 мегабайт}

Шахматная ассоциация решила оснастить всех своих сотрудников такими
телефонными номерами, которые бы набирались на кнопочном телефоне
ходом коня. Например, ходом коня набирается телефон \t{340-4927}. При
этом телефонный номер не может начинаться ни с цифры \t{0}, ни с цифры \t{8}.

Клавиатура телефона выглядит так:

\begin{tabular}{ l c r }
7 & 8 & 9 \\
4 & 5 & 6 \\
1 & 2 & 3 \\
& 0 & \\
\end{tabular}

Напишите программу, определяющую количество телефонных номеров 
длины $N$, набираемых ходом коня.

\InputFile
На входе записано целое число $N$ ($1 \le N \le 50$).

\OutputFile
Выведите искомое количество телефонных номеров.

\Example

\begin{example}
\exmp{2
}{16
}%
\end{example}

\end{problem}

