Сообщения SMS сотового телефона MOBILA составлены из прописных латинских букв.
Если буква первая на кнопке, нужно нажать эту кнопку один раз,
чтобы доСМСбавить букву в сообщение. Если буква вторая --- нужно нажать кнопку дважды и т.д.
Так, чтобы набрать слово \texttt{SMS}, нужно нажать \\
\texttt{(PQRS)(PQRS)(PQRS)(PQRS)(MNO)(PQRS)(PQRS)(PQRS)(PQRS)}

Клавиатура телефона выглядит так:

\begin{tabular}{ | l | c | r | }
\hline
& \t{ABC} & \t{DEF} \\ \hline
\t{GHI} & \t{JKL} & \t{MNO} \\ \hline
\t{PQRS} & \t{TUV} & \t{WXYZ} \\ \hline
\end{tabular}

Чтобы ввести две буквы, находящиеся на одной кнопке,
нужно между нажатиями клавиши сделать паузу.
Например, чтобы ввести сообщение \texttt{AA}, нужно нажать \\
\texttt{(ABC)(пауза)(ABC)}

Если на кнопке три буквы, то, как только такая кнопка нажата три раза,
последняя буква добавляется в сообщение немедленно, а следующие нажатия
той же кнопки относятся к следующей букве сообщения. Аналогично,
если на кнопке четыре буквы, то после четырёх нажатий в сообщение
будет добавлена последняя буква. То есть последовательность нажатий \\
\texttt{(ABC)(ABC)(ABC)(ABC)(пауза)(ABC)} \\
соответствует сообщению \texttt{CAA}.

К сожалению, сотовые телефоны этой модели давно не производятся,
и остался только один такой телефон. Он может произвольно вставлять
и игнорировать паузы во время ввода сообщения, что может привести к
некоторым изменениям в сообщениях. Например, введя \texttt{MOSCOWQUARTERFINAL},
можно получить вместо этого \texttt{OMSCMNWQTTARTERPDEINAL}.
Вы получили SMS-сообщение и знаете, что оригинальное сообщение содержало $N$ букв. Чтобы определить вероятность угадывания оригинального сообщения, найдите число возможных сообщений, которые могли превратиться в то, которое Вы получили.
