\begin{problem}{Минимальный путь в таблице}{стандартный ввод}{стандартный вывод}{1 секунда}{256 мегабайт}

В прямоугольной таблице $N \times M$ (в каждой клетке которой 
записано некоторое
число) в начале игрок находится в левой верхней клетке. 
За один ход ему
разрешается перемещаться в соседнюю клетку либо вправо, 
либо вниз.

При проходе через клетку с
игрока берут столько монет, какое число записано в 
этой клетке (деньги
берут также за первую и последнюю клетки его пути).

Требуется найти минимальную сумму, заплатив которую 
игрок может попасть в правый нижний угол.


\InputFile
Во входном файле задано два числа $N$ и $M$ --- размеры таблицы 
($1 < N < 20$, $1 < M < 20$). 
Затем идет $N$ строк по $M$ чисел в каждой ---
размеры штрафов за прохождение через 
соответствующие клетки (числа от $0$ до $100$).


\OutputFile
В выходной файл запишите минимальную сумму, 
потратив которую можно попасть в правый нижний угол.

\Example

\begin{example}
\exmp{3 4
1 1 1 1
5 2 2 100
9 4 2 1
}{8}%
\end{example}

\end{problem}

