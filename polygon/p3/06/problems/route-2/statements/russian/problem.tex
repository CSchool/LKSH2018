\begin{problem}{Маршрут 2}{стандартный ввод}{стандартный вывод}{1 секунда}{256 мегабайт}

\epigraph{Если задача ``Маршрут'' настолько хороша, то почему нет ``Маршрут 2''?}
{Никита}

Дана матрица $N \times N$, заполненная положительными числами. Путь по матрице начинается в левом верхнем углу. За один ход можно пройти в соседнюю по вертикали или горизонтали клетку (если она существует). Нельзя ходить по диагонали, нельзя оставаться на месте. Требуется найти максимальную сумму чисел, стоящих в клетках по пути длиной $K$ (клетку можно посещать несколько раз).

\InputFile
В первой строке находятся разделенные пробелом числа $N$ и $K$ ($2 \le N \le 100, 1 \le K \le 2000$). Затем идут N строк по N чисел в каждой, каждое число от 1 до 9999.

\OutputFile
Вывести одно число~---~максимальную сумму.

\Example

\begin{example}
\exmp{5 7
1 1 1 1 1
1 1 3 1 9
1 1 6 1 1
1 1 3 1 1
1 1 1 1 1
}{21}%
\end{example}

\end{problem}

