\begin{problem}{Квазибинарное число}{стандартный ввод}{стандартный вывод}{1 секунда}{256 мегабайт}

Назовём число квазибинарным, если его представление в десятичной системе счисления содержит только нули и единицы.

Сколько необходимо поменять цифр в данном числе, чтобы его можно было выразить, не больше чем $K$ квазибинарными слагаемыми?


\InputFile
Два целых числа: $N$, $K$ ($0 \le N \le 10^6$, $1 \le K \le 20$).


\OutputFile
Одно неотрицательное целое число - ответ на задачу.

\Example

\begin{example}
\exmp{42 3
}{1}%
\end{example}

\Note
$42$ можно представить как $10 + 10 + 11 + 11$ (4 слагаемых).

Если поменять четвёрку, например на единицу, то тогда $12$ можно будет представить как $11 + 1$ (2 слагаемых)


\end{problem}

