\begin{problem}{Задача про Павла Михайловича}{стандартный ввод}{стандартный вывод}{1 секунда}{256 мегабайт}

На занятия в старшую группу к Павлу Михайловичу ходят $N$ человек. Они решают задачи городских и областных олимпиад по программированию.

Однажды, решая сложную задачу на динамическое программирование, ребята попросили подсказку у Павла Михайловича. Он согласился дать подсказку только в том случае, если ребята обыграют его в такую игру. На доске написано число $K$. Каждый из учеников по очереди один раз прибавляет или вычитает единицу из числа на доске и записывает там результат. Если в результате они смогут получить ноль, Павел Михайлович даст им подсказку. Есть ли у ребят шанс?

\InputFile
Два числа $N$ и $K$, разделённые пробелом ($1 \le N, K \le 10^9$).

\OutputFile
Выведите <<Yes>>, если ребята смогут получить подсказку, или <<No>>, если нет.

\Example

\begin{example}
\exmp{9 9
}{Yes}%
\end{example}

\end{problem}

