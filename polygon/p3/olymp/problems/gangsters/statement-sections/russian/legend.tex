$N$ преподавателей Летней Компьютерной Школы ``Кэш'' собираются на ужин в столовую. $i$-й преподаватель приходит в момент времени $T_i$ и имеет несколько электронных баллов, а именно $P_i$. Дверь столовой имеет $K + 1$ степень открытости, они обозначаются целыми числами $0, 1, \ldots, K$. Сегодня в столовой дежурит Дима. Он маленький, а дверь тяжёлая, поэтому Дима может изменить степень открытости двери только на 1 за единицу времени, то есть в каждую единицу времени, он может приоткрыть дверь на 1, призакрыть дверь на 1 или оставить её в том же положении. В начальный момент дверь закрыта (степень открытости 0). Это сделано для того, чтобы осы не вылетали из столовой.

$i$-й преподаватель Летней Компьютерной Школы ``Кэш'' заходит в столовую, только если дверь открыта специально для него, то есть когда степень открытости двери равна его полноте $S_i$. Если же в момент, когда $i$-й преподаватель Летней Компьютерной Школы ``Кэш'' подходит к столовой, степень открытости двери не соответствует его полноте, то он уходит и больше не возвращается.

Столовая работает в интервале времени $\left[{}0, T\right]$. Диме для осуществления его хитрого плана необходимо собрать преподавателей Летней Компьютерной Школы ``Кэш'' с максимальным суммарным количеством электронных баллов, открывая и закрывая дверь в столовую соответствующим образом. Помогите ему.