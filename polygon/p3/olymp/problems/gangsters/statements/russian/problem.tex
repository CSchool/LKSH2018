\begin{problem}{Задача про ужин}{стандартный ввод}{стандартный вывод}{1 секунда}{256 мегабайт}

$N$ преподавателей Летней Компьютерной Школы ``Кэш'' собираются на ужин в столовую. $i$-й преподаватель приходит в момент времени $T_i$ и имеет несколько электронных баллов, а именно $P_i$. Дверь столовой имеет $K + 1$ степень открытости, они обозначаются целыми числами $0, 1, \ldots, K$. Сегодня в столовой дежурит Дима. Он маленький, а дверь тяжёлая, поэтому Дима может изменить степень открытости двери только на 1 за единицу времени, то есть в каждую единицу времени, он может приоткрыть дверь на 1, призакрыть дверь на 1 или оставить её в том же положении. В начальный момент дверь закрыта (степень открытости 0). Это сделано для того, чтобы осы не вылетали из столовой.

$i$-й преподаватель Летней Компьютерной Школы ``Кэш'' заходит в столовую, только если дверь открыта специально для него, то есть когда степень открытости двери равна его полноте $S_i$. Если же в момент, когда $i$-й преподаватель Летней Компьютерной Школы ``Кэш'' подходит к столовой, степень открытости двери не соответствует его полноте, то он уходит и больше не возвращается.

Столовая работает в интервале времени $\left[{}0, T\right]$. Диме для осуществления его хитрого плана необходимо собрать преподавателей Летней Компьютерной Школы ``Кэш'' с максимальным суммарным количеством электронных баллов, открывая и закрывая дверь в столовую соответствующим образом. Помогите ему.

\InputFile
В первой строке находятся числа $N$, $K$, $T$ ($1 \le N, K \le 100$, $1 \le T \le 30000$).

Во второй строке находятся числа $T_1, T_2, \ldots, T_N$ ($0 \le T_i \le T$).

В третьей строке находятся числа $P_1, P_2, \ldots, P_N$ ($1 \le P_i \le 300$).

В четвёртой строке находятся числа $S_1, S_2, \ldots, S_N$ ($1 \le S_i \le K$).

\OutputFile
Выведите одно число~---~максимальное суммарное количество электронных баллов преподавателей Летней Компьютерной Школы ``Кэш'', попавших в столовую на ужин. Если зайти не удалось никому. выведите \t{0}.

\Examples

\begin{example}
\exmp{4 10 20
10 16 8 16
10 11 15 1
10 7 1 8
}{26}%
\exmp{2 17 100
5 0
50 33
6 1
}{0}%
\end{example}

\end{problem}

