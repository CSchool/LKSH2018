\begin{problem}{Задача про Наталью Владимировну}{стандартный ввод}{стандартный вывод}{1 секунда}{256 мегабайт}

Наталья Владимировна на занятиях играла с детьми в такую игру. Каждый мог загадать трёхзначное число. Затем из него нужно было вычесть сумму его цифр, после чего отбросить любую из цифр результата.

После этого ученик называл две оставшиеся цифры, а Наталья Владимировна должна была угадать ту цифру, которую он отбросил. Почти всегда ей это удавалось.

Напишите программу, которая угадывает отброшенную цифру, если известны две оставшиеся.


\InputFile
Вводятся две десятичные цифры, разделённые пробелом.

\OutputFile
Выведите оставшуются цифру или <<Impossible>>, если её невозможно определить.

\Example

\begin{example}
\exmp{4 3
}{2}%
\end{example}

\end{problem}

