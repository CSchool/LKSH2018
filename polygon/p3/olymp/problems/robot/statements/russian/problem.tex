\begin{problem}{Задача про робота}{стандартный ввод}{стандартный вывод}{1 секунда}{256 мегабайт}

Робот Р-2008-2009 предназначен для исследования просторов Флатландии, которые, как известно, представляют собой части плоскости, разбитые на единичные квадраты (клетки) вертикальными и горизонтальными прямыми. Программы для этого робота достаточно просты, так как написаны
на языке программирования, который содержит всего четыре команды.

Эти команды таковы:
\begin{itemize}
\item сдвинуться на клетку вверх --- U
\item сдвинуться на клетку вниз --- D
\item сдвинуться на клетку влево --- L
\item сдвинуться на клетку вправо --- R
\end{itemize}

Ваша задача состоит в написании программы, которая будет анализировать некоторые свойства
программы для робота Р-2008-2009, --- а именно, предположим, что исследуемая область 
представляет собой бесконечную во всех четырех направлениях плоскость. Задана программа для робота
Р-2008-2009. Необходимо найти число клеток плоскости, которые он посетит более одного раза.

Заметим, что это число не зависит от того, в какой клетке изначально находится робот.


\InputFile
Единственная непустая строка содержит программу для робота. Она состоит только из символов
U, D, L, R. Ее длина не превосходит 1000 символов.


\OutputFile
Выведите ответ на задачу.


\Examples

\begin{example}
\exmp{ULDR
}{1}%
\exmp{URLD
}{2}%
\end{example}

\end{problem}

